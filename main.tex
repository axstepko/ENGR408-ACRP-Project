
\documentclass{article}
\usepackage{blindtext}
\usepackage{authblk}
\usepackage[a4paper, total={6.5in, 8in}]{geometry}
\usepackage{indentfirst}
\usepackage{setspace}
\usepackage{appendix}
\usepackage{multicol}

%Bibliography configuration:
\usepackage{biblatex}
\usepackage[english]{babel}
\addbibresource{bib2.bib}



\doublespacing %Global doublespace enable


\title{Sustainable Power Generation and Distribution for Electric Aircraft}
\author{
Michael Galiani | mpg5835@psu.edu\\
Brendan McLernan | bxm5554@psu.edu\\
Alexander Stepko | axstepko@psu.edu\\
Sana Tipnis | sat5652@psu.edu\\
Jaden Weed | jrw6156@psu.edu\\~\\
Advisor: Professor Steve Betza
}
\affil{The Pennsylvania State University}
\date{\today}


\renewcommand*\contentsname{Table of Contents}

\begin{document}             % End of preamble and beginning of text.
\begin{titlepage}
\maketitle                   % Produces the title.
\end{titlepage}

\tableofcontents %Table of contents for entire document, includes appendix
\newpage

\section{Executive Summary}\label{execSum}
\blindtext

\newpage

\section{Problem Statement and Background}\label{probStatement}

Current aviation infrastructure centers its  elf around a petroleum-based operational model.
This operational model, developed over decades by governmental agencies, airport authorities, and airlines has developed into a multi-billion-dollar industry responsible for the safe transportation of almost 3-million individuals every day in the United States alone (REF 19). 
Despite the massive success seen in  the airline industry and the overall increase in accessibility of air travel to the average citizen, there is now a major push by global governing bodies towards a sustainable future. 
Currently, the fuel we use produces various greenhouse gases that pollute more than just CO2, namely pollutants such as Nitrogen Oxides, Sulphur Dioxide, Soots, and various other microparticles (REF 11). 
These ‘non-CO2‘ pollutants, in addition to the CO2 emissions inherent in a hydrocarbon-based fuel, pose an adverse health risk to people in contact with it and the area surrounding the airport. 
Additionally, there is substantial evidence that exposure to common fuels such as Jet A and Jet A-1 likely increases people’s susceptibility to respiratory infections, asthma, and other chronic lung infections. (REF 12 and 20) . 
In addition to health risks, fuel spills pose an enormous environmental impact, and the chemistry and behavior of the fuel itself when released into the environment make it difficult to contain (REF 14, 15).\par



In addition to the adverse environmental effects that jet fuels pose, there are some operational issues to consider with our current aircraft support infrastructure. 
Commonly used fuels are kerosene-based, which is currently a derivative from fossil fuels. Due to this reliability on a single base hydrocarbon, supply chain and market instability has led to fuel shortages and price hikes. 
The result of this is an unstable industry revenue stream, whose burden is placed on the consumer (REF 13). 
In addition, electrical charging infrastructure is not widely in place and is much more costly to install compared to the existing fuel infrastructure (REF 16). 
The airline industry lacks penetration into affordable short-haul flights, which has immense potential for the future, but since CO2 emissions are not at breakeven, this remains widely unattainable using current methods. 
Current battery technology, although at a great disadvantage in power density compared to current fuel chemistry (W/Kg), provides a platform to operate short-haul aircraft on a smaller scale.\par


The environmental and operational impacts currently seen with the usage of traditional jet fuels, although successful in past decades, have demonstrated that there is a desire for change in how we provide energy to planes. 
Alternative energy sources are critical to power the planes of the future. Improving the materials and technologies that engines use today could lead to 70\% more fuel-efficient aircrafts and engines compared to 40 years ago.  (REF 21). 
The benefits of using sustainable aviation fuel (SAF) also extend beyond the environmental benefits. 
The usage of biomass-based fuel leads to improved aircraft performance and additional revenue for farms where the biomass was obtained (REF 18). 
Some of the sustainable alternative fuels currently in development show that these fuels can be 80\% less carbon-intensive  than traditional fossil-based jet fuels (REF 22). 
As a result, alternative fuel sources are inherently safer for the people and the environment and have the potential to make airport operations much more seamless and progressive. \par 


Despite the previous operational success seen with the traditional hydrocarbon-based energy model, the change to SAFs and Electric Aircraft (EA) changes must be made soon, as the environmental impacts are creating an unsustainable future for our planet. 
Not only do these changes affect the environment, but they pose a risk to outdated airport systems. 
Antiquated airports are unable to handle these new advancements in people and technology without substantial modification to the current operational model. 
The International Air Transport Association (IATA) reports a 123.2\% increase in jet fuel prices over the past 12 months (as of June 3rd) (REF 7). 
On commercial airliners, jet fuel accounts for about 40\% of the entire cost (REF 23).  
It is extremely costly to run these aircrafts and engines the way they are currently designed, which poses a major problem. 
With the major increase in discussion about our environmental impact on the planet, aircraft manufacturers have already begun the shift towards using SAFs in existing engines. 
This reduces the environmental burden slightly, but chemists and engineers are well on their way toward a completely sustainable future. 
Airports must do the same and be equipped to deal with this change in energy delivery. 
As mentioned earlier, Electric Aircraft (EA), an extremely plausible alternative energy aircraft, are becoming more popular. 
However, airports en masse currently lack the infrastructure to support them on a large flight network (REF 9). 
Our airports need to be equipped to move forward and handle these demands of modern technology. Therefore, it is imperative we start looking at how to make a change soon, to protect not only our planet but also, to make our airports more progressive to keep up with the ever-changing demands of a technologically advancing society.\par


The potential short-term impact of making these changes is largely a reduction in CO2 emissions. 
Airplanes make up 3\% of CO2 emissions but are the fastest-growing source of it (REF 6).  
There is an opportunity to grow the labor force as well. 
An additional set of labor is needed if multiple fuels are in wide use and more electricians will be needed on-site to troubleshoot issues with the widespread electric infrastructure necessary for EAs. 
Fire suppression techniques also will need to adapt to battery technology. 
The NFPA's standard 418 is working on techniques for this new sector (REF 17), so we will need a labor force to tackle these issues in both training and equipment. 
An airport that is prepared for a wide variety of sustainable aircraft will find itself nimbler during this transitionary period and will likely see greater investment from airlines. 
The result of this is plane manufacturers themselves have an incentive to invest, due to larger numbers of plane purchases using new technology (REF 6).\par 


It is also imperative to consider the transition to these future solutions. 
To accommodate an electric aircraft, the charging structure would need to be implementable at all airports as well as a means of producing most of the electricity to charge them (REF 4). 
The larger airports (~30 sites in the US) can utilize large solar arrays, but this will prove more difficult for the sma ller ones (5050 in US) (REF 4). 
Large infrastructure, such as biofuel-specific pipelines may be necessary to ensure carbon neutrality throughout all aspects of the fuel's life (REF 5). 
Fueling, although chemically different, will remain similar, but the volatility of new fuels may require mor e precise control of chemistry in storage – another potential hurdle airports must prepare for. 
There are several considerations to explore, and the transition will need to be feasible for airports across the U.S. to complete, so looking at the current obstacles in place and actively considering them in design is crucial. 
We expect planes to be retrofit to accommodate new fuels as well, making this transition much more feasible.\par


Our project aims to create new horizons for airports to easily adjust, encourage, and respond to future alternative powered aircrafts. 
We need to find a way for airports to create clean energy and cater to these alternate powered aircrafts, without hurting the environment consequently. 
We are aiming to create not only a solution for big public airports, but for the private sector as well.

\newpage

\section{Literature Review}\label{litRev}
 
\subsection{Introduction}\label{litIntro}
The airline industry's heavy reliance on a petroleum-based operational model poses significant challenges when trying to transition towards a sustainable, environmentally conscious, and yet still scalable business model. 
Currently, aircraft and aerospace industry manufacturers see two viable solutions for environmentally friendly air travel: electrically powered aircraft, and biofueled aircraft. 
The implementation of both solutions in the airline industry poses significant challenges; however, research explored in this publication demonstrates that these challenges can be overcome in both the electric and biofuel segments of sustainable air travel.\par
\subsection{The Impact of Electric Aircraft on Future Markets}
It is reasonable to expect that the transition towards electrically-powered planes will happen gradually, as is the case with many changes in air travel. 
There are significant barriers to implementing the electrification of an airport, namely a high initial investment period and long clearing time before profit is again seen (REF 24).  
Additionally, governmental oversight into the safety validation and operational stability of new electric systems further adds to transition timeframes (REF 24). 
This may discourage investment in Global Electric Aircraft Markets, yet a report conducted by Vantage Market Research shows favorable predictions. 
The 2021 valuation of this aircraft market was \$7.7b USD; in 2028, this valuation increases to \$17.8b USD. 
Other reports show potential market increase up to \$27.7b USD in the same year (2028) (REF 26). 
Predictions for widespread adoption of EA in the passenger transport / UAM sector would likely be between 2025 and 2035, given multiple variables including (but not limited to): market profiles, power infrastructure installations, utility support, permitting and approval processes, and construction times (REF 40).\par


This large increase in market share is emphasized in Asia-Pacific regions, however there is also substantial market growth available in the logistics industry (REF24). 
The advent of eCommerce shows that consumers demand products as fast as possible, regardless of the means by which it must arrive. 
The usage of smaller, electric planes for logistics has proven itself to be cost effective and efficient for parcel carriers (REF 27). 
DHL is an industry leader in logistics electrification, with hopes to create the “…world’s first electric air transport network.” (REF 27).\par


\subsection{Target Markets of Electric Aircraft and its Related Infrastructure}
There also exists potential demand in the urban air mobility (UAM) sector (REF 26), coinciding nicely with limitations related to power density of thoroughly-tested battery technology (REF 25 ). 
Aircraft used for this sector would likely be business jets, regional transport aircraft, and ultralight aircraft, as the smaller size of these machines provide a more favorable stance for electrification (REF 27). 
Private investors will see EA as more cost effective, as electric motor have proven to be more mechanically and financially efficient (REF 27).\par


The North American market accounts for the largest share of the EA industry at 34.3\% (REF 26). 
Favorably, the expected CAGR during the forecast period is 16.1\%. 
With these favorable market predictions, American Airlines has recently invested \$25m USD into Vertical Aerospace Group with the hopes of gaining penetration into UAM networks.  
A major hurdle to overcome in electrically powered aircraft is energy storage methods. 
Current battery technology is capable of power densities in the realm of 250  Wh/kg. 
Typical jet fuel has a power density of ~12000 Wh/kg (REF 25). 
For comparative energy storage, an EA with the same capacity would weigh approximately 30 times more than its hydrocarbon counterpart (REF 26).  
Advancements in battery technology from Lithium-polymer to Lithium-sulfur based chemistry may yield power density increases from 250 Wh/kg to ~500 Wh/kg , however this still pales in comparison to hydrocarbon-based fuels – the major hurdle to mass-electrification of an air transport network (REF 26). 
Consequently, current market predictions focus on logistic and UAM developments.\par


\subsection{Small-Scale Deployments of Electric Aircraft Infrstructure}
A crucial aspect of any airport operational model is energy transfer. 
Traditionally, this is accomplished via refueling with hydrocarbon fuels. 
Electric aircraft, obviously, use no hydrocarbons for their main propulsion system. 
Instead, the development of battery charging areas at the airport is necessary. 
As is expected of major system deployments, it is reasonable to expect EA infrastructure deployed on smaller air travel networks before implementation at large commercial sites.\par


Although not directly related to the topic of EA, the electrification of other airport support vehicles such as pedestrian busses and Ground Support Equipment (GSE) appears to have reasonably small barriers to deployment. 
The usage of electric buses in an urban setting has already shown success in both maintaining customer throughput rates and reducing the network’s impact on the environment.  
It is reasonable then to assume that implementing this infrastructure into an airport would hold similar results (REF 26). 
Complicated power management algorithms can be developed to maximize charging time under minimal electricity costs, which is generally during non-peak passenger throughput times (REF 26).\par


Manufacturers of EA infrastructure have already made great progress in small-scale deployments. 
US company Eaton’s Skycharge charging system claims great improvements to the charging experience of an EA. 
Among other improvements,  Eaton claims to have developed a solution capable of deployment to “any e-airplane, EVTOL [Electric Vertical Takeoff and Landing], or UAM aircraft" (REF 3). 
This inter-operability of multiple aircraft charging standards provides a large amount of flexibility to airports without having to consider the implementation of multiple charging standards. 
Being environmentally conscious, Skycharge’s system architecture also has framework to implement Eaton photovoltaic (PV) systems directly into the charger with ease (REF 3). 
The geographical requirements of an airport are such that a large, flat, open space is needed. 
Airports can be outfitted with PV systems (designed to eliminate pilot glare and obstruction), and then act as energy storage/distribution centers to not only their own infrastructure needs, but also to the grid (REF 26).\par


Smaller General Aviation airports have already begun to implement EA charging into their operational model. 
The Compton/Woodley Airport (FAA identifier CPM) in Los Angeles County, California has already integrated such a system into its existing energy-delivery network (REF 28). 
There has also been developments in Sweden for implanting charging infrastructure for fossil-free aviation. 
They set up three total charging stations, where two of them were set on the main apron of the airport and the other located in the southern part of the airport. 
This ensured aircrafts currently on the market the ability to utilize these charging stations, however, the duration of these potential flight times are unknown. 
The Bern-Belp airport in Switzerland utilizes the Eaton Skycharge system with good success (REF 3). 
Martha’s Vineyard Airport (FAA Identifier MVY) has also begun to investigate the implementation of EA charging into its operational model working alongside Beta Aircraft and Aviaton Technologies (REF 29). 
Sam Hobbs, of Beta, notes that chargers for their planes also work with, “…Teslas, Chevy Volts, and other electric vehicles.” (REF 29), demonstrating the flexible energy demands of light aircraft. 
Beta’s charging network is expansive, spanning much of the eastern seaboard and southeast United States (REF 30). The target market of this manufacturer is logistics and UAM, suggesting that reports produced by Vantage Markets Research (REF 24) and MarketsandMarkets Research Private Ltd. (REF 26) are accurate in their predictions. 
Similarly to Skycharge, their charging system emphasizes flexibility and ease-of-integration into existing power distribution channels (REF 30). 
For example, commercial aviation (including but not limited to scheduled, air taxi, and charter operations) fly with strict timetables. These aircraft require fast, on-demand charging likely during peak grid usage hours. 
General aviation is not subsequent to such strict timing demands, and therefore have more relaxed charging demands. 
These aircraft would be best to charge at non-peak electricity demand time periods. 
Regardless of the category of operation, demand management systems would be vital to ensure efficient operation (REF 40).\par


Of course, charging on-board batteries for EA is not the only potential solution. 
To greatly decrease turnaround time for an airframe, some concept designs feature capability to swap entire battery modules between planes and charging bays, eliminating the airframe’s downtime to charge batteries (REF 31). 
In addition to time savings, out-of-airframe charging with battery swaps ensures adequate cooling, charge rates, and the subsequent preservation the battery chemistry (REF 41). 
Amidst the great potential time and resource savings that swap-based infrastructure holds, it is unfortunately clear that the implementation of such a system is generally not feasable in many applications. 
A complicated analysis of a single airlne utilizing a single battery system shows major operational challenges when trying to manage prompt departures, arrivals, and battery management (REF 31). 
The conclusion of this research is that even at this incredibly small scale, there are enormous operational obstacles and inefficiencies that exist with the usage of battery swaps (REF 31). 
With the advent of new fast-charging batteries, the existing barriers with swap strategy become slightly more reasonable; however there still exists inherent advantages in utilizing on-airframe charging.  
The concept of utilizing a swap strategy is tantalizing, yet currently remains unreasonable for the foreseeable future (REF 32).\par



\newpage

\newpage

\section{Design Research Methodology}\label{designReview}
\blindtext

\section{Design Solution}\label{solution}
\blindtext 


%--------------Begin appendix section:----------------------
\newpage
\appendixpage
\appendix
\section{Contact Information}\label{apxA}

\noindent\textbf{Faculty Advisor(s):}\newline
% * <Alex Stepko> 13:34:06 01 Jul 2022 UTC-0400:
% steve or steven?
Steve Betza\\
Professor of Practice in Engineering Leadership\\
The Pennsylvania State University
\\~\\
\begin{multicols}{2}
\noindent\textbf{Student Information}\newline
Michael Galiani\\
Undergraduate Student in XXXXXXXXXXX\\
% * <Alex Stepko> 13:33:42 01 Jul 2022 UTC-0400:
% major
The Pennsylvania State University\\
mpg5835@psu.edu
\\~\\
Brendan McLernan\\
Undergraduate Student in XXXXXXXXXXXX\\
% * <Alex Stepko> 13:33:33 01 Jul 2022 UTC-0400:
% Need major here
The Pennsylvania State University\\
bxm5554@psu.edu
\\~\\
Jaden Weed\\
Undergraduate Student in XXXXXXXX\\
% * <Alex Stepko> 13:33:48 01 Jul 2022 UTC-0400:
% major
The Pennsylvania State University\\
jrw6156@psu.edu\columnbreak
\\~\\
Alexander Stepko\\
Undergraduate Student in Electrical Engineering\\
The Pennsylvania State University, Schreyer Honors College\\
axstepko@psu.edu
\\~\\
Sana Tipnis\\
Undergraduate Student in XXXXXXXXXXXX\\
% * <Alex Stepko> 13:33:53 01 Jul 2022 UTC-0400:
% major
The Pennsylvania State University\\
sat5652@psu.edu
\end{multicols}
\newpage


\section{Description of University}\label{apxB}
\blindtext[3]

\newpage
\section{Non-University Partner Research Biographies}\label{apxC}
\begin{singlespace}
\noindent\textbf{Steve Betza} is a former corporate director of the Future Enterprise Initiative at Lockheed Martin. 
He served as an adviser to the Department of Defense, National Security Agency, and Defense Science Board, helping with electronic, U.S. industrial base, and manufacturing innovation issues. 
Betza led teams in developing advanced flight computers with IBM and held several executive titles at Lockheed Martin as a key figure in billion-dollar companies. 
He is also presently a Professor of Practice in Engineering Leadership at The Pennsylvania State University, making him a great candidate to interview for the purposes of that class project. 
Betza opened our eyes and broadened our horizons for our research and prototyping process. 
Our team learned most planes used today still have design frames from the '50-'60s, only really being updated internally. 
This means the option for swapping aircraft internals with alternate propulsion systems is extremely viable. 
From this, we concluded the transition from the planes we know and fly today to E-planes would be relatively seamless in-plane structure and flight behavior aspects. 
Our meeting with Betza withheld countless strides of improvement in our research and design process, but perhaps the most important takeaway was that the process of alternating aviation infrastructure into electric-based will be a long and difficult path; but nevertheless, completely attainable and maybe even natural.\\~\\
\noindent\textbf{Bill Grauer} is a former test engineer for Boeing's wind tunnel, vibration, and propulsion research facilities.
He currently consults for Boeing’s Philadelphia site and has over 40 years of experience in the aircraft manufacturing industry. 
Mr. Grauer provided technical insight into aircraft hybridization, biofuels' viability as propulsion energy, and the technical hurdles associated with an electric propulsion system. 
Being a propulsion test engineer, there was visibility provided into the usage of biofuels in existing or modified engine architecture. 
Additionally, a brief touch on the usage of rotary-winged aircraft for UAM shows that enormous technical hurdles remain in this sector. 
Mr. Grauer’s involvement in the industry over a broad time period shows there is limited viability in the use of biofuels in existing or new airframes, and that electrically based propulsion is likely to be more well-received by -the airline industry, airport operators, and governmental bodies. 

\end{singlespace}

\section{Sign-off form for faculty advisor(s) and/or department chair(s)}\label{apxD}
\blindtext
%----Begin bibliography section:---------------------------
\section{References}
\printbibliography[heading = none]
\cite{fuelSpecs}

\end{document}               % End of document.